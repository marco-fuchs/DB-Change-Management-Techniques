% % % % % % % % % % % % % % % % % % % % % % % % % % % % % % % % % % % % % % % % %
%
% LaTeX Documentation Template - Header File
%
% % % % % % % % % % % % % % % % % % % % % % % % % % % % % % % % % % % % % % % % %
% "THE BEER-WARE LICENSE" (Revision 42):
% Hannes Badertscher (hbaderts@hsr.ch) wrote this file. As long as you retain 
% this notice you can do whatever you want with this stuff. If we meet some day, 
% and you think this stuff is worth it, you can buy me a beer in return. 
% - Hannes Badertscher
% % % % % % % % % % % % % % % % % % % % % % % % % % % % % % % % % % % % % % % % %

% % % % % % % % % % % % % %
% Basic document setup\usepackage[T1]{fontenc}

\documentclass[
    11pt,                  % Font size
    final,
    parskip=half,          % Half a line skipped between paragraphs
    twoside,               % Two-sided document
    openright,             % Chapters start on right pages
    bibliography=totoc,    % Bibliography in ToC
    listof=totoc,          % LoF and LoT in ToC
]{scrreprt}[2015/09/15]    % Use current version of KOMA-Script


% Author and title of document
\author{\Author}
\title{\Title}
\date{\today}

% Load Fonts
%\usepackage[T1]{fontenc}                      % Use T1 encoding
%\usepackage[scaled=0.85]{berasans}            % Sans-serif: Bera Sans
%\usepackage[scaled=0.84]{beramono}            % Mono-space: Bera Mono
%\usepackage[sc]{mathpazo}                     % Serif: Palatino
%\linespread{1.05}                             % More linespread for Palatino

\usepackage{helvet}
\renewcommand{\familydefault}{\sfdefault}

% Define text area
\typearea[10mm]{9}
\setlength{\marginparwidth}{25mm}
\textheight = 644pt
% Basic packages
\usepackage[final,activate={true,nocompatibility}]{microtype}    % Enable micro-typography
\usepackage{scrhack}             % Fixes koma-script incompatibilities
\usepackage{mparhack}            % Improved marginpar placement
\usepackage{float}				 % Precise Image Placement

% Load HSR colors

%marco -----------------

% underbraces
\usepackage{mathtools}

% checkmark
%\usepackage{MnSymbol,utfsym,arev}
\usepackage{dingbat}

\newcommand\pro{\item[$+$]}
\newcommand\con{\item[$-$]}

\usepackage{lipsum} 

\usepackage[ruled,vlined]{algorithm2e}

\usepackage{enumitem}

\usepackage[table,xcdraw]{xcolor}
\usepackage{siunitx}
\usepackage{lscape} % querformat

\usepackage{trfsigns} % Transformation Symbol o---o \laplace and \Laplace

%\usepackage[fleqn]{amsmath}

\newcommand{\vlaplace}[1][]{\mbox{\setlength{\unitlength}{0.1em}%
        \begin{picture}(10,20)%
            \put(3,2){\circle{4}}%
            \put(3,4){\line(0,1){12}}%
            \put(3,18){\circle*{4}}%
            \put(10,7){#1}
        \end{picture}%
    }%
}%

\newcommand{\vLaplace}[1][]{\mbox{\setlength{\unitlength}{0.1em}%
        \begin{picture}(10,20)%
            \put(3,2){\circle*{4}}%
            \put(3,4){\line(0,1){12}}%
            \put(3,18){\circle{4}}%
            \put(10,7){#1}
        \end{picture}%
    }%
}%                     



\usepackage{enumitem,amssymb}
\newlist{todolist}{itemize}{2}
\setlist[todolist]{label=$\square$}
\usepackage{pifont}
\newcommand{\cmark}{\ding{51}}%
\newcommand{\xmark}{\ding{55}}%
\newcommand{\done}{\rlap{$\square$}{\raisebox{2pt}{\large\hspace{1pt}\cmark}}%
    \hspace{-2.5pt}}
\newcommand{\wontfix}{\rlap{$\square$}{\large\hspace{1pt}\xmark}}

% % % % % % % % % % % % % %
\usepackage{xcolor}
\usepackage{header/HSRColors}
\usepackage{soul}


%Marco
\newcommand{\zerotext}[2][0pt]{\makebox[#1][l]{\qquad#2}}
\newcommand{\hlc}[2][yellow]{{%
            \colorlet{foo}{#1}%
            \sethlcolor{foo}\hl{#2}}%
}

\usepackage[a4paper]{geometry}

%---

% Header and footer
\usepackage[automark,headwidth=textwithmarginpar,footwidth=text,headsepline=0.4pt:textwithmarginpar]{scrlayer-scrpage}  % Custom header and footer
\pagestyle{scrheadings}
\renewcommand{\headfont}{\normalfont\sffamily}
\makeatletter
\renewcommand{\chaptermark}[1]{\markboth{\@chapapp~\thechapter~--~#1}{}}
\makeatother
\rohead{\rightmark}
\lehead{\leftmark}

% Bibliography
\usepackage[backend=biber,style=ieee]{biblatex}
\renewbibmacro*{bbx:savehash}{}                       % Don't abbreviate identical authors
\defbibheading{bibintoc}[\bibname]{%                  % Make bibliography title a chapter*
    \chapter*{#1}\markboth{#1}{#1}%
    \addcontentsline{toc}{chapter}{#1}%
}

% Glossary
\usepackage[toc]{glossaries}
\renewcommand{\glsnamefont}[1]{\makefirstuc{#1}}

% Index
\usepackage{imakeidx}
\makeindex[intoc,columnseprule]
\indexsetup{firstpagestyle=plain}    % Show header/footer on index page

% Aligned footnotes
\usepackage[hang]{footmisc}
\setlength{\footnotemargin}{1em}

% % % % % % % % % % % % % %
% Language stuff
\newcommand*{\LangDE}{de}
\ifx \Lang \LangDE
    \usepackage[english,ngerman]{babel}  % Main language: German
\else
    \usepackage[ngerman,english]{babel}  % Main language: English
\fi

% % % % % % % % % % % % % %
% Additional packages

% Date and time format
\usepackage{datetime}
\newdateformat{titledate}{\THEDAY.~\monthname\space\THEYEAR}

% Allow multicolumn document
\usepackage{multicol}

% Math
\usepackage{amsmath}
\usepackage{amssymb}
\usepackage{bm}

% Tables
\usepackage{multirow}
\usepackage{tabularx}
\usepackage{booktabs}

% Figures
\usepackage{pdfpages}
\usepackage{epstopdf}

% Listings - if you need advanced listings, switch to minted!
% Listing styles by by github.com/rnestler / github.com/dkoeppel for github.com/HSR-Stud/header
% Licensed under CC-BY-NC-SA

\ifx\GUARDlistings\undefined		% include guard
\def\GUARDlistings{}

\usepackage{listings}

	\lstdefinestyle{Java}{ numbers=left,
	  belowcaptionskip=1\baselineskip,
	  breaklines=true,
	  frame=L,
	  xleftmargin=\parindent,
	  language=Java,
	  showstringspaces=false,
	  basicstyle=\footnotesize\ttfamily,
	  keywordstyle=\bfseries\color{green!40!black},
	  commentstyle=\itshape\color{purple!40!black},
	  identifierstyle=\color{blue},
	  stringstyle=\color{orange},
	  numberstyle=\ttfamily\tiny
	}
	
	\lstdefinestyle{SQL}{
	  numbers=none,
	  belowcaptionskip=1\baselineskip,
	  breaklines=true,
	  xleftmargin=\parindent,
	  language=SQL,
	  showstringspaces=false,
	  basicstyle=\footnotesize\ttfamily,
	  keywordstyle=\bfseries\color{green!40!black},
	  commentstyle=\itshape\color{purple!40!black},
	  identifierstyle=\color{blue},
	  stringstyle=\color{orange},
	}
	
	\lstdefinestyle{C}{
	  numbers=left,
	  belowcaptionskip=1\baselineskip,
	  breaklines=true,
	  frame=L,
	  xleftmargin=\parindent,
	  language=C,
	  showstringspaces=false,
	  basicstyle=\footnotesize\ttfamily,
	  keywordstyle=\bfseries\color{green!40!black},
	  commentstyle=\itshape\color{purple!40!black},
	  identifierstyle=\color{blue},
	  stringstyle=\color{orange},
	  numberstyle=\ttfamily\tiny
	}
	
	\lstdefinestyle{Cpp}{
	  numbers=left,
	  belowcaptionskip=1\baselineskip,
	  breaklines=true,
	  frame=L,
	  xleftmargin=\parindent,
	  language=C++,
	  showstringspaces=false,
	  basicstyle=\footnotesize\ttfamily,
	  keywordstyle=\bfseries\color{green!40!black},
	  commentstyle=\itshape\color{purple!40!black},
	  identifierstyle=\color{blue},
	  stringstyle=\color{orange},
	  numberstyle=\ttfamily\tiny
	}
	
	\lstdefinestyle{Csharp}{
	  numbers=left,
	  belowcaptionskip=1\baselineskip,
	  breaklines=true,
	  frame=L,
	  xleftmargin=\parindent,
	  language=[Sharp]C,
	  showstringspaces=false,
	  basicstyle=\footnotesize\ttfamily,
	  keywordstyle=\bfseries\color{green!40!black},
	  commentstyle=\itshape\color{purple!40!black},
	  identifierstyle=\color{blue},
	  stringstyle=\color{orange},
	  numberstyle=\ttfamily\tiny
	}
	
	
	\lstdefinestyle{Matlab}{
	  numbers=left,
	  belowcaptionskip=1\baselineskip,
	  breaklines=true,
	  frame=L,
	  xleftmargin=\parindent,
	  language=Matlab,
	  showstringspaces=false,
	  basicstyle=\footnotesize\ttfamily,
	  keywordstyle=\bfseries\color{green!40!black},
	  commentstyle=\itshape\color{purple!40!black},
	  identifierstyle=\color{blue},
	  stringstyle=\color{orange},
	  numberstyle=\ttfamily\tiny
	}
	
	\definecolor{mygreen}{rgb}{0,0.6,0}
	\definecolor{mygray}{rgb}{0.5,0.5,0.5}
	\definecolor{mymauve}{rgb}{0.58,0,0.82}
	\definecolor{grund}{gray}{0.9} 

	
\lstset{ 
  backgroundcolor=\color{grund},   % choose the background color; you must add \usepackage{color} or \usepackage{xcolor}; should come as last argument
  basicstyle=\footnotesize,        % the size of the fonts that are used for the code
  breakatwhitespace=true,         % sets if automatic breaks should only happen at whitespace
  breaklines=true,                 % sets automatic line breaking
  captionpos=b,                    % sets the caption-position to bottom
  commentstyle=\color{mygreen},    % comment style
  deletekeywords={...},            % if you want to delete keywords from the given language
  escapeinside={\%*}{*)},          % if you want to add LaTeX within your code
  extendedchars=true,              % lets you use non-ASCII characters; for 8-bits encodings only, does not work with UTF-8
  firstnumber=1,                		% start line enumeration with line 1000
  frame=single,	                   % adds a frame around the code
  keepspaces=false,                 % keeps spaces in text, useful for keeping indentation of code (possibly needs columns=flexible)
  keywordstyle=\color{blue},       % keyword style
  language=Python,                 		% the language of the code
  morekeywords={*,...},            % if you want to add more keywords to the set
  numbers=left,                    % where to put the line-numbers; possible values are (none, left, right)
  numbersep=8pt,                   % how far the line-numbers are from the code
  numberstyle=\tiny\color{mygray}, % the style that is used for the line-numbers
  rulecolor=\color{black},         % if not set, the frame-color may be changed on line-breaks within not-black text (e.g. comments (green here))
  showspaces=false,                % show spaces everywhere adding particular underscores; it overrides 'showstringspaces'
  showstringspaces=false,          % underline spaces within strings only
  showtabs=false,                  % show tabs within strings adding particular underscores
  stepnumber=2,                    % the step between two line-numbers. If it's 1, each line will be numbered
  stringstyle=\color{mymauve},     % string literal style
  tabsize=2	                   % sets default tabsize to 2 spaces
  %title=\lstname                   % show the filename of files included with \lstinputlisting; also try caption instead of title
}
	
	


\fi
% \usepackage[newfloat, chapter]{minted}
% \usemintedstyle{friendly}

% Si Units
\usepackage{siunitx}
\sisetup{detect-all,sticky-per,per-mode=symbol}

% Quotation marks
\usepackage{csquotes}
\setquotestyle[quotes]{german}

%Picutures
%\usepackage{graphicx}

% % % % % % % % % % % % % %
% Numbering and captions
\numberwithin{equation}{chapter}
\numberwithin{figure}{chapter}
\numberwithin{table}{chapter}

% Setup captions
\usepackage{caption}[2008/08/24]
\usepackage{subcaption}
\setkomafont{captionlabel}{\scshape\color{HSRBlue}}
\captionsetup{labelsep=quad}

% Setup toc and section numbering depth
\setcounter{tocdepth}{2}
\setcounter{secnumdepth}{4}

%marco
%\usepackage[table,xcdraw]{xcolor}
% % % % % % % % % % % % % %
% Hyperref setup
\usepackage[
    pdftitle={\Title},
    pdfauthor={\AuthorA \AuthorB},
    pdfkeywords={\Keywords},
    pdflang={\Lang},
    pdfpagemode=UseOutlines,  % Show outlines when opening pdf.
    pdfdisplaydoctitle=true,  % Show document title in pdf viewer.
    pdfcreator={LaTeX with hyperref and KOMA-Script},
    colorlinks=true,
    linkcolor=HSRBlue,
    citecolor=HSRBlue,
    filecolor=HSRBlue,
    urlcolor=HSRBlue,
    bookmarksnumbered=true
]{hyperref}



% If \Print=true, then make all links black for nicer print
\providecommand*{\True}{true}
\ifx \Print \True
    \hypersetup{hidelinks}
\fi

% Nameref setup
\usepackage{zref-titleref}
\makeatletter
\newcommand*{\currentname}{\zref@getcurrent{title}}
\makeatother


% Set reference names
\addto\extrasenglish{% English
    \renewcommand*{\figureautorefname}{Fig.}             % fig:
    \renewcommand*{\tableautorefname}{Tab.}              % tab:
    \renewcommand*{\equationautorefname}{Eq.}            % eq:
    \renewcommand*{\chapterautorefname}{Chp.}            % chp:
    \renewcommand*{\sectionautorefname}{Sec.}            % sec:
    \renewcommand*{\subsectionautorefname}{Sec.}         % subsec:
    \providecommand*{\listingautorefname}{Listing}
}
\addto\extrasngerman{% German
    \renewcommand*{\figureautorefname}{Abb.}             % fig:
    \renewcommand*{\tableautorefname}{Tab.}              % tab:
    \renewcommand*{\equationautorefname}{Gl.}            % eq:
    \renewcommand*{\chapterautorefname}{Kapitel}            % chp:
    \renewcommand*{\sectionautorefname}{Kapitel}            % sec:
    \renewcommand*{\subsectionautorefname}{Kapitel}         % subsec:
    \renewcommand*{\subsubsectionautorefname}{Kapitel}      % subsubsec:
}

% % % % % % % % % % % % % %
% Marginpar setup
\usepackage{ragged2e}
\newcommand*{\oldmarginpar}{}
\let\oldmarginpar\marginpar
\renewcommand*{\marginpar}[1]{%
    \leavevmode\oldmarginpar%
    [\RaggedLeft\scshape\footnotesize\textcolor{HSRBlue}{\hspace{0pt}#1}]%
    {\RaggedRight\scshape\footnotesize\textcolor{HSRBlue}{\hspace{0pt}#1}}%
}

% % % % % % % % % % % % % %
% Itemize items
\newcommand{\hsrlistitemi}{\textcolor{HSRBlue}{\raisebox{.3ex}{\tiny$\blacksquare$}}}
\newcommand{\hsrlistitemii}{\textcolor{HSRLightGray}{\raisebox{.3ex}{\tiny$\blacksquare$}}}
\newcommand{\hsrlistitemiii}{\textcolor{HSRLightGray}{\raisebox{.3ex}{\tiny$\blacktriangleright$}}}

\renewcommand{\labelitemi}{\hsrlistitemi}
\renewcommand{\labelitemii}{\hsrlistitemii}
\renewcommand{\labelitemiii}{\hsrlistitemiii}

% % % % % % % % % % % % % %
% Chapter headings
\makeatletter
% In Document titles
\renewcommand{\@makechapterhead}[1]{%
    \vspace*{10\p@}%
    \hfill%
    \begin{minipage}[b]{9cm}%
        \raggedleft%
        \sffamily\huge\textbf{\thechapter}
        \sffamily\huge\textbf{#1}%

    \end{minipage}%
    \quad%
    %    {\fontsize{60pt}{3em}\selectfont\sffamily\textbf{\textcolor{HSRBlue}{\thechapter}}}%
    \vskip 5\p@%
    \noindent\makebox[\textwidth-2em][l]{\textcolor{HSRBlue}{\rule{\paperwidth-\oddsidemargin-\hoffset-1in}{0.7pt}}}%
    \vskip 25\p@%
    \normalfont\normalsize%
}

\renewcommand{\@makeschapterhead}[1]{%
    \vspace*{10\p@}%
    \hfill%
    \begin{minipage}[b]{9cm}%
        \raggedleft%
        \sffamily\huge\textbf{#1}%
    \end{minipage}%
    \vskip 5\p@%
    \noindent\makebox[\textwidth-2em][l]{\textcolor{HSRBlue}{\rule{\paperwidth-\oddsidemargin-\hoffset-1in}{0.7pt}}}%
    \vskip 25\p@%
    \normalfont\normalsize%
}
\makeatother

% % % % % % % % % % % % % %
% Appendix page + toc entry
\usepackage{appendix}

\makeatletter
%\renewcommand{\@chap@pppage}{%
%    \clear@ppage
%    \thispagestyle{empty} % this was 'plain' before
%    \if@twocolumn\onecolumn\@tempswatrue\else\@tempswafalse\fi
%    \null\vfil
%    \markboth{}{}%
%    {%
%        \centering
%        \interlinepenalty \@M
%        \normalfont
%        %\Huge \bfseries\sffamily \appendixpagename\par % this was rmfamily 
%        \Huge \bfseries\sffamily Anhang \par % this was rmfamily
%    }%
%    \if@dotoc@pp
%        \addappheadtotoc
%    \fi
%    \vfil\newpage
%    \if@twoside
%        \if@openright
%            \null
%            \thispagestyle{empty}%
%            \newpage
%        \fi
%    \fi
%    \if@tempswa
%        \twocolumn
%    \fi
%}
\makeatother

% % % % % % % % % % % % % %
% TikZ and PGF Plots
\usepackage{tikz}
\usepackage{pgfplots}

% Tikz Libraries
\usetikzlibrary{positioning}

% Use Sans-Serif family in TikZ graphics
\tikzset{font=\sffamily}

% If there are many TikZ images, make them external.
% Note: you'll have to call pdflatex with the -shell-escape flag!
%\usetikzlibrary{external}
%\tikzexternalize[prefix=tikz/]

% Create cycle list with HSR colors
\pgfplotscreateplotcyclelist{hsrcolorlist}{%
    HSRSchwarz,every mark/.append style={solid,fill=HSRSchwarz}     \\%
    HSRBlue,every mark/.append style={solid,fill=HSRBlue}           \\%
    HSRHematite,every mark/.append style={solid,fill=HSRHematite}   \\%
    HSRBasswood,every mark/.append style={solid,fill=HSRBasswood}   \\%
}
% Set default cycle list for plots
\pgfplotsset{cycle multi list={mark list\nextlist hsrcolorlist}}

% Set default cycle list for bar plots
\pgfplotsset{
    /pgfplots/bar cycle list/.style={/pgfplots/cycle list={%
                    {HSRBlue,fill=HSRBlue60,mark=none},%
                    {HSRHematite,fill=HSRHematite60,mark=none},%
                    {HSRBasswood,fill=HSRBasswood60,mark=none},%
                }}
}

% % % % % % % % % % % % % %
% ToDo notes
\newcommand{\todonotecolor}{red}

% Small TODO note
\newcommand{\todonote}[1]{%
    \marginpar{\textcolor{\todonotecolor}{ToDo}}%
    \colorbox{red}{\sffamily\textcolor{white}{#1}}%
    \PackageWarning{ToDo:}{#1!}%
}

% Large TODO note
\newcommand{\todo}[1]{%
    \marginpar{\textcolor{\todonotecolor}{ToDo}}%
    \colorbox{red}{\parbox{\linewidth}{\sffamily\textcolor{white}{#1}}}%
    \PackageWarning{ToDo:}{#1!}%
}

% Command: Citation needed!
\newcommand{\citationneeded}{%
    \marginpar{\textcolor{\todonotecolor}{ToDo}}%
    \sffamily\textcolor{red}{$^\text{[Citation needed]}$}
    \PackageWarning{ToDo:}{Citation needed!}%
}

\usepackage{calc}
\usepackage{color}
\definecolor{mygreen}{rgb}{0,0.6,0}
\definecolor{mygray}{rgb}{0.5,0.5,0.5}
\definecolor{mymauve}{rgb}{0.58,0,0.82}
\definecolor{grund}{gray}{0.9}

\newcommand{\env}[1]{\texttt{#1}}
\newcommand{\command}[1]{\texttt{#1}}
\newcommand{\package}[1]{\texttt{\itshape#1}}

\usepackage{gensymb}
\usepackage{cleveref}
