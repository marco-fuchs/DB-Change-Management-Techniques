% !TeX spellcheck = en_GB
%\cleardoublepage
\markleft{\abstractname}
\pdfbookmark[1]{Abstract}{abstract}

\chapter*{Abstract}
\marginpar{Introduction}%
Database change management is a crucial aspect of database administration that ensures changes made to a database are consistent, reliable, and reversible. This is achieved through version control, change request processes, thorough testing, and comprehensive documentation, as already done in software development. The goal of database change management is to minimize the risk of data loss or corruption while maintaining the integrity and reliability of the data stored in the database. It also standardizes the process of making changes to the database and makes it easier to manage, and reduces the likelihood of errors and conflicts. Sooner or later, all database systems must adapt their schema design, which is best done automatically to avoid long downtimes. However, this DevOps of databases or DataOps is often forgotten and creates a development gap. 

\marginpar{Methods}%
This seminar project investigates and compares two commonly used database change management tools: Flyway and Liquibase. A literature review of database change management techniques was conducted. Furthermore, the tools Flyway and Liquibase were compared to each other by implementing six different change scenarios.

\marginpar{Results}%
The analysis shows that both tools provide the basic functionality needed for database change management. The main difference is that Flyway integrates changes by taking SQL files as input, while Liquibase needs a changelog file with defined change types.

\marginpar{Discussion}%
We cannot conclude that one tool is better than the other. Moreover, each tool has its strengths and shortcomings. Therefore, the choice of tool boils down to the team's individual needs and the database system's requirements.

