%!TeX spellcheck = en

\chapter{Possible MSE Data Engineering Exercise}

\section{Exercise: Database Change Management}
Goal of this exercise define and push different changes on the Pagila database using Liquibase.

\subsection*{Task 1}
\marginpar{Task 1}
Setup the Pagila database and install Liquibase onto your system using following links. Additionally, initialise a project and create a SQL changelog file.

Pagila database: \url{https://github.com/devrimgunduz/pagila}

Liquibase: \url{https://docs.liquibase.com/start/install/home.html}

Tip: Use the command \texttt{liquibase} to see all available commands.

\subsection*{Task 2}
\marginpar{Task 2}%
Define a changeset in the changelog file that renames the column "email" to "private\_email" in the customer table. The changeset should include the change and a corresponding rollback statement. Update the database and test the rollback. Use the \texttt{update} and \texttt{rollback} command.

\subsection*{Task 3}
\marginpar{Task 3}%
Setup a second project and use a database agnostic format for the changelog file like XML, JSON or YAML. Make sure to rollback any changes before working with the new changelog. Reimplement Task 2 using the new format. 

Available change types: \url{https://docs.liquibase.com/change-types/home.html}


\section{Solution}

\subsection*{Task 2}
\marginpar{Solution Task 2}%
Define the change using sql and the changeset format provided by Liquibase. SQL changelog cannot perform auto rollbacks and need to be defined by the user.

\begin{lstlisting}[language=SQL, caption={Sample Solution Task 2}]
	--changeset StudentName:1 labels:Task2 
	--comment: Task 2, Renaming an Attribute
	
	ALTER TABLE customer RENAME COLUMN email TO private_email;
	
	--rollback ALTER TABLE customer RENAME COLUMN private_email TO email;
\end{lstlisting}

\subsection*{Task 3}
\marginpar{Solution Task 3}%
Liquibase provides a change type that is able to rename columns. The change type comes with auto rollback 

\begin{lstlisting}[language=SQL, caption={Sample Solution Task 3}]
	- changeSet:  
		id:  1  
		author:  StudentName  
		changes:  
		-  renameColumn:  
			catalogName:  customer  
			columnDataType:  text  
			newColumnName:  private_email  
			oldColumnName:  email  
			schemaName:  public  
			tableName:  customer
\end{lstlisting}

