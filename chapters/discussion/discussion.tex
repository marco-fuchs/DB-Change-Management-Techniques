%!TeX spellcheck = en

\chapter{Results}

\section{Discussion}
\marginpar{Change Management}%
Key takeaways from Piairo et al. 2018 \cite{Piairo2018}:

\begin{itemize}
	\item Databases, although different from applications, can and should be included in the same development process as applications. We call this database shift left.
	\item When all database changes are described with scripts, source control can be applied, and the database development process can include continuous integration and continuous delivery activities, namely taking part in the deployment pipeline.
	\item Automation is the special ingredient that powers up the deployment pipeline making it repeatable, reliable and fast, reducing fear of database changes.
	\item Migrations-based and state-based are two different approaches to describing a database change. Independently of choice, small batches are always a good choice.
	\item The deployment pipeline is a technical and cultural tool where DevOps values and practices should be reflected according to the needs of each organisation.    
\end{itemize}

\marginpar{Flyway vs Liquibase}%
Flyway and Liquibase are open-source database change management tools that help organisations track, version, and deploy changes to their databases. Both tools support various database platforms and offer version control, change tracking, and rollback capabilities.

One key difference between the two tools is how changes are defined. Flyway uses SQL scripts to define changes, while Liquibase allows changes to be defined using XML, YAML, or JSON files. This means that Liquibase provides an abstraction layer that can make it easier to manage changes in environments where the underlying database technology may vary.

Another difference is how changes are applied. Flyway applies changes in a linear, version-based sequence, while Liquibase allows for more flexible change ordering and dependencies. This can make Liquibase better suited for complex change management scenarios but may also make it more challenging for simple deployments.

Overall, the choice between Flyway and Liquibase will depend on an organisation's specific needs and constraints. However, both tools are widely used and have strong communities of users and contributors, so either tool may be a good fit for a given project.

\section{Lesson learned}
The project teaches about the importance of database change management and the integration of data ops into the workflows of database management. Learning the state-of-the-art methods and exploring the developed tools such as Liquibase and Flyway is refreshing. We learned how easily Liquibase and Flyway work and how to use them to manage a database and gained insights on the best practices by using the tools ourselves. Database schema management is for more than just big teams with massive databases. Even small teams can benefit from the vast features and the version control it allows.









