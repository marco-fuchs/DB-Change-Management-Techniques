
% !TeX spellcheck = en_US

\chapter{Introduction to Flyway}


\section{Overview}
\marginpar{Information}%
Flyway is a tool that makes database migrations easy and provides version control for your database. It is a multi-platform and cross-database tool with support for over 20 databases.
Axel Fontaine from Google Code created Flyway in 2010, and acquired by Redgate Software by 2019. The creator was looking for a simple method to implement database changes easily and directly in SQL. His main goal was, to include database changes as part of the software deployment process \cite{Robles2021}.
Flyway uses the freemium business model.  The basic version is free and open source \cite{Fontaine2010} and additionally there is a paid Teams or Enterprise Edition available.\\
With flyway you can either write migration files with SQL (database-specific syntax (such as PL/SQL, T-SQL, etc.) or Java (for advanced data transformations or dealing with LOBs). Flyway is able to connect to databases hosted in either on-premises or cloud environments, thanks to the included JDBC driver library that is shipped with the tool. It runs on all modern platforms such as Linux, Mac or Windows.  \cite{Dillon2022}, \cite{DBMSTools}


\marginpar{Migrations \cite{Parsick2018}}%
There are four possibilities of migrations as shown in \autoref{tab:migration_types}. There are versioned migrations with scripts that have a unique meaning, which are executed only once. These versioned migrations are mainly used for Data Definition Language (DDL) like a create or modify a table. \\

Repeatable migration scripts do not have a version number. These are always if their checksum changes and always after the versioned migrations are applied. Typical applications are import base data or (re)-create views or functions.

\begin{table}[h]
	\centering
	\begin{tabularx}{8cm}{X|c c}
		& Versioned & Repeatable \\ \hline
		SQL based & \checkmark & \checkmark \\
		Java based & \checkmark & \checkmark \\
	\end{tabularx}
	\caption{Migration Types - Based on \cite{Parsick2018}}
	\label{tab:migration_types}
\end{table}

%\begin{center}
%\begin{tabularx}{8cm}{X|c c}
%	& Versioned & Repeatable \\ \hline
%SQL based & \checkmark & \checkmark \\
%Java based & \checkmark & \checkmark \\
%\end{tabularx}
%\end{center}


\marginpar{SQL migrations}%
Flyway developers can migrate changes directly with SQL. Per default, flyway takes the database migrations from the directory \textit{sql}.
These SQL migration files have to follow the naming convention:\\

\begin{center}
\begin{tabularx}{8cm}{r l}
Versioned & \texttt{\textcolor{blue}{V}\textcolor{green}{1\_1\_1}\textcolor{red}{\_\_}\textcolor{orange}{create\_table}.sql}\\
Repeatable & \texttt{\textcolor{blue}{R}\textcolor{red}{\_\_}\textcolor{orange}{create\_table}.sql}\\
\end{tabularx}
\end{center}

\begin{flushright}
\textcolor{blue}{Prefix}\\
\textcolor{green}{Version}\\
\textcolor{red}{Seperator (two underscores)}\\
\textcolor{orange}{Description}\\
\end{flushright}

These SQL scripts can have several rows and allowed is all database specific syntax including comments.

\marginpar{Java migrations}%

\marginpar{Features}%
\todo{Soll das Folgende in den Anhang?}

In the following paragraph, a few very useful flyway commands are presented:\\

\textbf{Info}\\
Get an overview of the applied migrations and their success status.\\

\textbf{Repair}\\
When a database migration fails, the migration is markes as failed in the schema history table (\textit{flyway\_schema\_history}). If a database supports DDL \footnote{Data Definition Language} transactions (like PostgreSQL), the failed migration is rolled back automatically and nothing is recorded in the schema history table.  But if the database does not support DDL transactions (e.g. Oracle Database, MySQL, MariaDB), you have two options to repair the database and remove the failed entries:

\begin{enumerate}
	\item Run \texttt{flyway repair}\\
	\item There are callbacks supported by flyway. One could use the \textit{afterMigrateError} and add a SQL-script to this callback, which deletes the failed migration entry in the \textit{flyway\_schema\_history} table.
	
	\begin{lstlisting}[language=SQL]
		DELETE IGNORE FROM flyway_schema_history WHERE success=0;
	\end{lstlisting}
\end{enumerate}

\textbf{clean}\\
With \texttt{flyway clean} one can drop all objects like tables, views, triggers, etc. in the configures schemas.It can easily give you a fresh start but is dangerous in a production environment.

\textbf{Validate}\\
If one want to be sure that all migrations are applied correctly,  \texttt{flyway validate} is the command of choice. The validation checks if the migration locally still has the same checksum as the migration executed in the database.
There is a possibility to apply custom validation rules. Because in a productive environment, there will be hotfixes, deleted migrations and other changes that break the default validation conventions (Flyway Teams Edition only).

\textbf{Undo}\\
To undo the most recent migration applied to the database, run \texttt{flyway undo}. The undo command can be repeated until the database is converted back to version 1.
Undo assumes that the previous applied migration succeeded and now should be undone. If the previsous applied migration failed, first repair the migrations before applying the undo-command. 
The undo-command is a Teams Edition feature only.

\textbf{Baseline}\\
This command makes the current database as the baseline for the future. This will cause Migrate to ignore all migration up to the actual version.
This can be useful to reduce the overhead, if you have many old migrations scripts that will not be used anymore.

\marginpar{Flyway Teams \cite{FlywayTeams}}%
The Pro-Version of Flyway is called Flyway Teams. This pro version costs (as of creation of this work) 447€ per user per year. It offers:
\begin{itemize}
	\item Additional migration controls
	\item Protect against failed deployments
	\item Professional echnical support from Redgate
	\item Support for older DB versions
	\item Built-in Git client and object-level versioning
\end{itemize}

In comparison to the team version, the free open source version includes the core functionalities with a Desktop GUI support for only the current DB versions and support through community. The core functionalities include the six basic commands: Migrate, Clean, Info, Validate, Baseline and Repair.

\marginpar{Community}%
Stars on GitHub: 6'800 (31. October 2022)\\
Tags in StackOverflow: 2,088 (31. October 2022)\\

\marginpar{Learning\\ Materials}%
If you have never worked with Flyway before there are the following ways to educate yourself:
\begin{itemize}
	\item \href{https://www.red-gate.com/hub/university/courses/flyway}{Redgate University Flyway training courses}
	\item \href{https://flywaydb.org/documentation}{Get Started Documentation}
	\item \href{https://www.youtube.com/playlist?list=PLhFdCK734P8DYHYYWaJpzJJ-qZFZ_JTHM}{Redgate Youtube Channel}
	\item \href{https://www.youtube.com/watch?v=dzRzlDpdDW4}{Flyway talk by Sandra Parsick}
\end{itemize}

\section{Installation and Setup}
Download the latest version of the \href{https://flywaydb.org/download/community}{Flyway Cummunity Edition} and extract the downloaded file. To execute the flyway commands, one need a java installation.
Once extracted, the file becomes a directory with the following structure:

\begin{figure}[H]
    \centering
    \includegraphics[width=0.4\textwidth]{./chapters/intro_flyway/images/flyway_folder_structure}
   \caption[Flyway folder structure - Source: Own illustration]{Flyway folder structure}
    \label{fig:flyway_folder_structure}
\end{figure}


\marginpar{Configuration}%
To connect to a running database, add the relevant information to \textit{conf/flyway.conf}.
In our minimal example this could be:

\begin{lstlisting}[caption=Minimal configuration]
flyway.url=jdbc:postgresql://localhost:5432/pagila
flyway.user=postgres
flyway.password=password
\end{lstlisting}

This is just a minimal configuration, but there are many more additional parameters that can be set \footnote{\url{https://flywaydb.org/documentation/configuration/parameters/}}.


\marginpar{Workaround Mac OS}%
To make the flayway command know on MacOS:
\begin{lstlisting}[caption=Minimal configuration]
export PATH=$PATH:/Users/marco/Documents/DB-Seminar/flyway-9.8.1
\end{lstlisting}
Running a flyway command for the first time, the error message \textit{„java“ kann nicht geöffnet werden, da der Entwickler nicht verifiziert werden kann.} occurs. Go to Security in settings and allow the java instance to run.

\newpage